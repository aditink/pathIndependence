\newcount\draft\draft=1 % set to 0 for submission or publication
\documentclass[sigplan,review,anonymous]{acmart}
\setcopyright{none}
%
\usepackage{tikz}
\usepackage{amssymb}
\usepackage[norelsize, section]{algorithm2e}
\usepackage{xxx}
%
\usepackage{graphicx}
% Used for displaying a sample figure. If possible, figure files should
% be included in EPS format.
%
% If you use the hyperref package, please uncomment the following line
% to display URLs in blue roman font according to Springer's eBook style:
% \renewcommand\UrlFont{\color{blue}\rmfamily}
\AtBeginDocument{%
  \providecommand\BibTeX{{%
    \normalfont B\kern-0.5em{\scshape i\kern-0.25em b}\kern-0.8em\TeX}}}

\begin{document}
\settopmatter{printacmref=false}
\title{Online Verification of Commutativity}
%
%\titlerunning{Abbreviated paper title}
% If the paper title is too long for the running head, you can set
% an abbreviated paper title here
%
\author{First Author}
\email{email}
\author{Second Author}
\authornotemark[1]
\email{email 2}
\affiliation{%
  \institution{Cornell University}
  \streetaddress{address}
  \city{Ithaca}
  \state{New York}
  \postcode{14850}
}

\author{Third Author}
\email{email 3}

\begin{abstract}
Commutative diagrams arise in many systems, such as an implicit type conversion graph in some programming languages.
It can be useful in program analysis to verify that a diagram representing conversions remains commutative over the course of its construction.
A naive approach to solving the problem would take factorial time, and the presence of cycles could potentially lead to an infinite number of paths to check. Previous work provides an algorithm for the special case of acyclic diagrams that verifies that a given diagram commutes in $O(|V|^4|E|^2)$ time, where V is the set of vertices in the diagram, and E, the set of edges. 
We present an algorithm that given a diagram and an edge to add to it, verifies that the resultant diagram commutes in $O(|V|(|E| + |V|)$ time. For the case when checking the equality of paths is expensive, we present an optimization that runs in $O(V^4)$ time but reduces to the minimum possible number of equality checks. We evaluate the algorithms against baselines, and see its practical application in the compiler of a domain specific language for geometry types. As a demonstration of the more general applicability of the algorithm, we use it to identify discrepancies in a  currency conversion graph.
\end{abstract}
%
%
%

\maketitle

\section{Introduction}
Many systems can be understood as graphs: a set of objects with directed functions between members. With programming, for example, type systems with coercions can be drawn as a graph with type nodes and coercion edges.  A desirable trait of such a type system and many other similar systems is that traversing the graph from a given start and end node results in the same value independent of path chosen; a property equivalent to saying the graph diagram is commutative.  With our type coercion example, it could be problematic if casting to a supposedly equivalent type as an intermediate step resulted in a different answer than a direct cast.

Given the utility of system diagram commutativity, it would be helpful to check that the property holds for real systems.
Since diagrams may change over time, and verifying the entire system from scratch may be computationally expensive, it would be additionally convenient if new edges could be checked online as they are added.

Naively checking if all paths in a given graph are path independent could require a number of function equality checks as bad as factorial in the number of nodes, since a path consists of an ordering of nodes. Previous work ~\cite{commutative} has identified an $O(|E|^2|V|^4)$ algorithm to verify that a complete acyclic diagram commutes; however, this solution neither manages the batch case nor the existence of cycles. Unfortunately, the presence of cycles implies a potentially infinite number of paths. 

We present a polynomial time $O(|V|(|E|+|V|)$ algorithm to verify that commutativity of a graph (potentially with cycles) is maintained over the course of online addition, assuming an oracle to check the abstract notion of equality of functions. We show the number of calls to the equality checking oracle is asymptotically tight. Should the calls to the oracle be expensive, we also show, with an additional $O(|V|^4)$ optimization step, an algorithm to use instead the minimal possible number of equality checks.

To evaluate our results we use two case studies.
First, our algorithm is used for the domain specific geometry type language \textit{Gator} to ensure that user defined transformations between spaces stay consistent. Second, our algorithm is used to identify inefficiencies in a currency conversion graph.

We compare our solution to three baseline implementations: a naive checking of all path pairs with only special handling for cycles, a check for all path pairs that involve the new edge, and an algorithm suggested by previous work to solve the batch version of the problem for acyclic diagrams.

\section{Formal Problem Setup and Terminology}

We start by formalizing the notion of a diagram, drawing inspiration from the previous acyclic work by Murota ~\cite{commutative}.

\xxx[AK]{A lot of this exposition is from the cited paper, I feel like I need to make this clearer to give proper credit, but not sure how}

We start with a directed graph G=(V,E), where V consists of sets of elements and edges (u, v) in E correspond to functions that maps elements of u to elements in v.
All these functions form a semigroup F, where multiplication is function composition.
A semigroup consists of a set and an associative binary operation, which we use to capture function composition.

The correspondence between edges and functions is stored as a mapping f:$E\rightarrow F$, where f maps each edge to the function it represents.

A path is a sequence of edges. The edge-to-function mapping f can be naturally extended to paths: if path p=$e_1\circ e_2 ... e_n$ then $f(p)=f(e_1) \circ f(e_2) ... f(e_n)$.

Let $P_{all}$ be the set of all paths in G.\xxx[dg]{Do we need this notation?  I don't see it anywhere.  Delete stuff that doesn't show up in the rest of the paper}

The notation $\partial^{+}(p)$ is used to indicate the start node of path p, and similarly $\partial^-(p)$ is the end node of path p. \xxx[dg]{Same for this notation.  I see $\partial$ being useful, but not $\partial^-$} $\partial(p)$ maps to the tuple (start node of p, end node of p).

A pair of paths p and q is called \textit{parallel} iff their terminal nodes are the same, i.e., $\partial(p)=\partial(q)$.
Let $R_{all}$ be the set of all parallel pairs of paths in the diagram.

The diagram commutes iff f(p)=f(q) $\forall (p,q)\in R_{all}$; that is, the composition of maps along any path connecting u to v is independent of path choice. Here, equality of functions is defined by some oracle whose specifics will vary by target functions and domain.

\section{Baseline Algorithms}

To examine the efficacy of our proposed algorithm, we compare it to some potential alternatives.  Specifically, we examine a na\"{i}ve factorial algorithm, which we identify to be a two-flip tolerant path search, and Murota's historical batch solution~\cite{commutative}.

\subsection{Na\"{i}ve algorithm for online addition}

We start with a reference baseline solution for online addition, where the original diagram is path independent and a new edge is added. Pairs that do not involve the new edge must continue to be path independent because they already belonged to the old diagram.   Any implementation can thus verify only parallel pairs that have been affected by the addition of the new edge.

That the original diagram commutes can be used to narrow down the set of pairs to verify.  Cases of pairs where either both paths or neither path includes the new edge must both remain equal with straightforward reasoning.

The cycle checking steps can also be improved upon. In the case where the identity is assumed, only elementary cycles passing through the new edge need to be verified to be equal to the identity. When the identity is not assumed, cycles that include one of the nodes of the new edge must be checked for the new exit or entry point that the new edge becomes (if the cycle includes the source of the new edge then the cycle followed by the new edge must be equal to the new edge alone and if the cycle include the sink then the new edge followed by the cycle must be equal to the new edge).
Additionally, all cycles that include the new edge would need full verification like in the batch algorithm- the cycle would have to be equal to itself composed with itself, and would have to be verified for all its entry and exit points.\xxx[dg]{This paragraph needs to be rewritten to make cycles clear without the identity stuff.  I'm not sure how to do it; I'll come back to it later}

Thus we are left only to verify the pairs where exactly one of the paths includes the new edge.

\paragraph{Two flip tolerant path search}
We use a "two-flip tolerant" path search to identify the pairs of paths where exactly one path includes the new edge. 

In a normal directed graph path, only forward edges, i.e. edges that go outward from the current node, are considered. 
A \textit{two flip path} consists of up to three phases: in the first phase, only backward edges- pointing inward to the current node- are accepted. In the second phase, only forward edges are accepted, and in the third phase, again only backward edges are accepted. The node between the first two phases we refer to as the \textit{first flipping point}, which has both edges pointing outward; similarly, we refer to the node between the latter two phases as the \textit{second flipping point}, at which both edges point inwards.

To present this idea diagrammatically, we represent path phases with arrows (note that these are the composition of many edges, not a single edge). The new edge is represented with a dashed arrow.

\begin{center}
\begin{tikzpicture}[scale=0.2]
\tikzstyle{every node}+=[inner sep=0pt]
\draw [black] (18.1,-36.1) circle (3);
\draw (18.1,-36.1) node {$S$};
\draw [black] (50.2,-36.1) circle (3);
\draw (50.2,-36.1) node {$T$};
\draw [black] (26,-22.9) circle (3);
\draw (26,-22.9) node {$F_1$};
\draw [black] (41.3,-22.9) circle (3);
\draw (41.3,-22.9) node {$F_2$};
\draw [dashed] (21.1,-36.1) -- (47.2,-36.1);
\fill [black] (47.2,-36.1) -- (46.4,-35.6) -- (46.4,-36.6);
\draw (34.15,-35.6) node [above] {new edge};
\draw [black] (24.46,-25.47) -- (19.64,-33.53);
\fill [black] (19.64,-33.53) -- (20.48,-33.1) -- (19.62,-32.58);
\draw [black] (29,-22.9) -- (38.3,-22.9);
\fill [black] (38.3,-22.9) -- (37.5,-22.4) -- (37.5,-23.4);
\draw [black] (48.52,-33.61) -- (42.98,-25.39);
\fill [black] (42.98,-25.39) -- (43.01,-26.33) -- (43.84,-25.77);
\end{tikzpicture}
\end{center}
Here, (S, $F_1$, $F_2$, T) is a two flip tolerant path. ($F_1$, S, T, $F_2$) is a new path, created because of the addition of (S,T), that conflicts with ($F_1$, $F_2$).

The two flip tolerant path search returns the set of all paths between a given source and sink that have up to two flips (paths that omit one or more of the three phases are also accepted).

The \textit{path extraction algorithm} then transforms the output of the two flip path search to the set of new pairs of parallel paths that arise in the diagram due to its addition, that need to be verified.

\paragraph{Path extraction algorithm}
The purpose of this algorithm is to derive the conflicting pairs that a given two-flip tolerant path corresponds to.

Let the new edge added to the diagram be (S, T).
It has already been shown how each path corresponds to a conflicting pair in the case where a path has two flips. When a path has only the first flip (which is to say, the third phase of the path is missing), but not the second flip, sink node T can be treated as the sink of the two conflicting path, while the first flipping point remains the source. Similarly when only the second flipping point is present then it is the sink, and S is the source. Finally when no flipping points are present, there are two possibilities. 
Either the path is oriented from S to T, in which case the conflicting paths are simply the edge (S, T) and the entire flip-less path, or the path is oriented from T to S. In this case, we have found a cycle, which has already been handled.

\paragraph{Theorem} Consider the result of two-flip tolerant path search from the source to sink node of an edge that is to be newly added followed with the path extraction algorithm described.  This result has a one-to-one correspondence to the set of new parallel pairs with exactly one path passing through the new edge and neither paths containing any cycles.

\paragraph{Proof}

Every element in the output of the path extraction algorithm was by construction a conflicting pair.

It remains to show that every new conflicting pair corresponds to a two flip tolerant path. Let the common source be $F_1$ and common sink be $F_2$. 

Only one path passes through (S, T), which we will name path 1. We construct a two flip tolerant path from S to T: phase 1 is the segment of path 1 from $F_1$ to S, phase 2 is path 2, and phase 3 is the segment of path 1 from T to $F_2$. It is possible that some of $F_1, F_2, S$ and $T$ coincide, in which case the corresponding segments between them disappear, and the resultant path has fewer than two flips.

We have shown that all paths that need verification are caught by the two flip tolerant search followed with path extraction.

\paragraph{Implementation}
\xxx[AK]{Copy in python code from implementation? Appendix?}
\xxx[DG]{Not sure we need this paragraph, the implementation is pretty clear from the description of the algorithm}

This algorithm is also in the worst case as bad as factorial in the number of nodes. However, in the average case, it significantly outperforms the naive batch baseline. For sparse diagrams it is even competitive with our ultimate polynomial solution. Average case heuristic analysis will be presented in the evaluation section.

\subsection{Optimal batch solution}
Murota's main result ~\cite{commutative} is a solution to the question of whether a given acyclic diagram commutes that returns the minimal number of equality checks. The paper describes the following algorithm to find the ($V^2E$ bounded) minimal set of pairs that needs to be checked.

A diagram's structure can result in some redundancy, in the sense that verifying a subset of pairs can imply the verification of path pairs that aren't in the subset.
For example, in the figure below, verifying $p_1$ and $p_2$ implies that the paths in $p_3$ are equal too.

\begin{center}
\begin{tikzpicture}[scale=0.2]
\tikzstyle{every node}+=[inner sep=0pt]
\draw [black] (12,-15.8) circle (3);
\draw [black] (30.5,-30.5) circle (3);
\draw [black] (46.4,-15.8) circle (3);
\draw [black] (29.9,-15.8) circle (3);
\draw [black] (15,-15.8) -- (26.9,-15.8);
\fill [black] (26.9,-15.8) -- (26.1,-15.3) -- (26.1,-16.3);
\draw (20.95,-16.3) node [below] {$f_1$};
\draw [black] (14.35,-17.67) -- (28.15,-28.63);
\fill [black] (28.15,-28.63) -- (27.84,-27.74) -- (27.21,-28.53);
\draw (19.94,-23.64) node [below] {$g_1$};
\draw [black] (29.063,-27.873) arc (-157.62406:-197.70132:13.638);
\fill [black] (29.06,-27.87) -- (29.22,-26.94) -- (28.3,-27.32);
\draw (27.49,-23.25) node [left] {$f_2$};
\draw [black] (31.53,-18.309) arc (25.84176:-21.16715:11.992);
\fill [black] (31.53,-18.31) -- (31.43,-19.25) -- (32.33,-18.81);
\draw (33.28,-23.03) node [right] {$j_1$};
\draw [black] (32.9,-15.8) -- (43.4,-15.8);
\fill [black] (43.4,-15.8) -- (42.6,-15.3) -- (42.6,-16.3);
\draw (38.15,-16.3) node [below] {$j_2$};
\draw [black] (32.7,-28.46) -- (44.2,-17.84);
\fill [black] (44.2,-17.84) -- (43.27,-18.01) -- (43.95,-18.75);
\draw (39.77,-23.64) node [below] {$h_1$};
\end{tikzpicture}

$p_1$=($f_1;f_2$, $g_1$)\\
$p_2$=($j_1;j_2$, $h_1$)\\
$p_3$=($f_1;h_1$, $f_1;j_2$)
\end{center}

\xxx[dg]{Should be a figure}
The approach in this algorithm, at a high level, is to define a function that takes in a subset of pairs and returns the subset of pairs whose verification is implied by verifying the input set. Then, greedy elimination of redundancies is performed until a minimal set is reached.

A bilinking is defined to be a parallel pair that is disjoint but for their terminal nodes. The set of all bilinkings is $R_0$.
In an acyclic diagram, if all bilinkings are equal, all parallel pairs must also be equal since they any given pair can be expressed as a composition of bilinkings.

Define $r_1>r_2$ for bilinkings $r_1$ = $\{p_l,q_l\}$, $r_2$= $\{p_2, q_2)$ $\in$ $R_0$, if there exists a path p such that $\partial p=\partial r_1$ and p contains $p_2$.
Define $\langle\rangle$ as:
$\langle r \rangle = \{ s\in R_0| r>s\}$.

For bilinking $s$, let $F(s)$ be the vector in $\mathbb{F}2^{|E|}$ representing the edges present in s (the $n^{th}$ dimension of $F(s)$ is 1 if the corresponding edge is in $s$, and 0 otherwise). Let this function be extended to sets, so that for some set of bilinkings $S, F(S) = \{ F(s) | s\in S \}$.

for a set of bilinkings $r$, the closure function $cl$ is defined as:
\xxx[dg]{I think this definition is incomplete}
$cl(r) = \{ s\in R_0| s$ is linearly dependent on $F(r) \}$.

The closure function on $r$ basically captures all the pairs that can be made by made by composing or "gluing together" the bilinkings in r. 

Using these two function we finally define the function $\sigma$ on a set of bilinkings r as
$\sigma(r) = \{s \in R_0 | s\in cl(R\cap \langle s \rangle) \}$.
This is the function used to capture all the pairs whose verification is implied by the verification of pairs in r.

$\sigma$ is used to iteratively check if a given pair is redundant. Bilinkings are eliminated until a minimum "spanning" subset is reached.

A complete implementation of the described algorithm is included in the appendix.

\xxx[dg]{Probably put this in the appendix}
\xxx[dg]{Give a summary of the algorithm though, breaking down important parts.  I think that's mostly been done above, but I'm not sure}
\begin{verbatim}
Graph existingGraph;
R_s = {}
for each node v in V:
    # Let the subsection of the graph reachable from v be S.
    S = existingGraph.extractReachableSection(v)
    # Create a minimum spanning tree of S
    T = createMinimumSpanningTree(S)
    # Find the excluded edges
    excludedEdges = S.edges - T.edges
    # Use these edges to create bilinkings
    for each edge e in excludedEdges:
        firstPath = T.findPath(source: e.source, sink: e.sink)
        R_s.addElement(new Bilinking(firstPath, e)
return R_s
\end{verbatim}

\begin{verbatim}
Find a spanning set Rs, = [r_1, ... ,r_k].

R := Rs,
    for i:=l to K do
        if r_i in sigma(R - r_i) then R:=R - r_i;
return R.

sigma(inputSet S, codomain = Rs):
    output = {}
    for bilinking in Rs:
        smallerPairs = {}
        for candidateBilinking in Rs:
            if exists path(source: bilinking.source, sink:candidateBilinking.source) and
            exists path(source: candidateBilinking.sink, sink: bilinking.sink):
                smallerPairs.add(candidateBilinking)
        # Check if there's a linear dependence between the bilinking and smallerPairs
        # vectorize(pair) returns a vector with a dimension for each edge in the graph,
        # with a 1 if pair includes that edge and 0 if it doesn't.
        matrix = [vectorize(pair) for pair in smallerPair]
        matrix = matrix+[vectorize(bilinking)]
        # if there is a linear dependence
        if determinant(matrix)%2==0:
            output.add(bilinking)
    return output
                
\end{verbatim}

The proof of correctness can be found in Murota\cite{commutative}, 
The number of checks returned by the algorithm is at worst $O(|V|^2|E|)$. The overall run time of an optimized implementation is $O(|V|^4|E|^2)$.

\section{Solving the online addition problem}

The solution presented here assumes that there is an implicit identity function on every node, as would be the case in most applications. However, a modified version of the algorithm for diagrams without this assumption is presented in the appendix. The modification is similar in spirit to what was described in the description of the naive baselines.

Like with the online baseline, we do not concern ourselves with parallel pairs where neither or both paths pass through the new edge.
The key observation that allows us to improve on the online baseline is that for a given (source, sink) pair only a single parallel pair needs to be verified. This is an implication of Theorem \ref{reductionRule}, expanded on later.
Theorem \ref{verifyingSet} shows that should our selected set of pairs along with cycles passing through the new edge be verified commutative, the entire diagram must commute. The approach is to identify a parallel pair with exactly one path through the edge for each (source, sink) pair.


\begin{verbatim}
Graph existingGraph;
Edge newEdge;

Set parallelPairs = new Set();
for S in existingGraph.Nodes():
    for T in existingGraph.Nodes():
        try:
            Path pathWithNewEdge = 
                FindPath(
                    graph: existingGraph, 
                    sourceNode: S,
                    sinkNode: newEdge.Source()) +
                newEdge as Path +
                FindPath(
                    graph: existingGraph, 
                    sourceNode: newEdge.Sink(), 
                    sinkNode: T)
            if (S==T):
                // Cycles are a speacial case.
                pathInOldGraph = identity(S)
            else:
                Path pathInOldGraph = FindPath(
                    graph: existingGraph, 
                    sourceNode: S, 
                    sinkNode: T);
            parallelPairs.add((pathInOldGraph, pathWithNewEdge));
        catch (pathFindingFailed Exception):
            // No comparable pairs from node {S} to node {T} 
            // that need to be checked
            continue

output parallelPairs;
\end{verbatim}
\begin{algorithm}
    \caption{Polynomial time algorithm}
    \label{polynomial}
    \KwData{placeholder}
    \KwResult{placeholder}
\end{algorithm}

The try block is executed at most $O(|V|^2)$ times, so that this is the bound on the number of pairs verified.

The bound is asymptotically tight. This can be seen in the case where the graph contains 2N nodes besides S and T. We consider N of the nodes to be in group 1, and the other N to be in group 2. Every node in group 1 has a forward node to every node in group 2, as well as to S. T has a forward edge to every node in group 2. In this diagram, on the addition of edge (S, T), $N^2$ paths need to be verified which is polynomial in 2N+2.

Also notice that if trying to optimize for path length (say, if composing functions is expensive) then "find any path" can be replaced with "find shortest path".

An efficient implementation of the algorithm can run in $O(|V|(|V|+|E|))$ time at a $O(|V|^3)$ space cost. In such an implementation, path finding from a given source node to all potential sink nodes would be done in a single $O(|V|+|E|)$ breadth first search and the results memoized in $O(|V|^2)$ space.

\subsection{Optimization Step}

In the case where equality checks are very expensive, we begin by finding the minimal set of (source, sink) pairs such that checking for these pairs logically implies having checked the full diagram.

We observe that there are some redundancies in the diagram.

Consider the following situation:

\begin{center}
\begin{tikzpicture}[scale=0.2]
\tikzstyle{every node}+=[inner sep=0pt]
\draw [black] (16.6,-34.1) circle (3);
\draw (16.6,-34.1) node {$S$};
\draw [black] (56.8,-33.6) circle (3);
\draw (56.8,-33.6) node {$T$};
\draw [black] (28,-20.1) circle (3);
\draw (28,-20.1) node {$P_1$};
\draw [black] (44.5,-20.1) circle (3);
\draw (44.5,-20.1) node {$P_2$};
\draw [black] (28,-44.9) circle (3);
\draw (28,-44.9) node {$Q_1$};
\draw [black] (44.5,-44.9) circle (3);
\draw (44.5,-44.9) node {$Q_2$};
\draw [black] (31,-20.1) -- (41.5,-20.1);
\fill [black] (41.5,-20.1) -- (40.7,-19.6) -- (40.7,-20.6);
\draw (36.25,-20.6) node [below] {$g_1$};
\draw [black] (26.11,-22.43) -- (18.49,-31.77);
\fill [black] (18.49,-31.77) -- (19.39,-31.47) -- (18.61,-30.84);
\draw (21.74,-25.67) node [left] {$f_1$};
\draw [black] (25.82,-42.84) -- (18.78,-36.16);
\fill [black] (18.78,-36.16) -- (19.01,-37.08) -- (19.7,-36.35);
\draw (23.57,-39.02) node [above] {$f_2$};
\draw [black] (31,-44.9) -- (41.5,-44.9);
\fill [black] (41.5,-44.9) -- (40.7,-44.4) -- (40.7,-45.4);
\draw (36.25,-45.4) node [below] {$g_2$};
\draw [black] (54.59,-35.63) -- (46.71,-42.87);
\fill [black] (46.71,-42.87) -- (47.64,-42.7) -- (46.96,-41.96);
\draw (49.21,-38.76) node [above] {$h_2$};
\draw [black] (54.78,-31.38) -- (46.52,-22.32);
\fill [black] (46.52,-22.32) -- (46.69,-23.25) -- (47.43,-22.57);
\draw (51.19,-25.39) node [right] {$h_1$};
\draw [black] (28,-23.1) -- (28,-41.9);
\fill [black] (28,-41.9) -- (28.5,-41.1) -- (27.5,-41.1);
\draw (27.5,-32.5) node [left] {$l$};
\draw [black] (19.6,-34.06) -- (53.8,-33.64);
\fill [black] (53.8,-33.64) -- (52.99,-33.15) -- (53.01,-34.15);
\draw [black] (44.5,-41.9) -- (44.5,-23.1);
\fill [black] (44.5,-23.1) -- (44,-23.9) -- (45,-23.9);
\draw (45,-32.5) node [right] {$m$};
\end{tikzpicture}
Each arrow represents a path, and (S,T) is the new edge being added.
\end{center}
\begin{theorem}
\label{reductionRule}
If conflicting paths $g_2 = f_2; (S,T); h_2$ then it must be that $g_1 = f_1; (S,T); h_1$.
\end{theorem} 
% monomorphisms
\begin{proof}
We use the fact that $f_1$=$l; f_2$ and $h_1$=$h_2; m$.
\[g_2 = f_2; (S,T); h_2 \Rightarrow l; g_2 = l; f_2; (S,T) ; h_2 \]
\[\Rightarrow l ; g_2 ; m = l ; f_2 ; (S,T) ; h_2 ; m \Rightarrow g_1 = f_1 ; (S,T) ; h_1\]
\end{proof}

Note that the proof is not affected if any of these paths is the identity, eg. if $f_1$ is the identity and S and $P_1$ are actually the same node.

We conclude that verifying a comparable pair of paths with end points ($P_1$, $P_2$) implies the verification of all path pairs ($Q_1$, $Q_2$) such that $Q_1$ is a successor of $P_1$ and $P_2$ is a successor of $Q_2$. A successor S to node N is any node such that there exists a path from N to S. Nodes are also their own successors and predecessors.

Under the assumption that the only path operations allowed are composition and replacement of one path by a different, equal path, as would be true when edges are generic functions, and no other information is available, so that F is a semi-group, this "reduction" rule is also the only rule to reduce the set of path pairs to check by finding implications.

That is to say, if verifying a comparable pair of paths with end points ($P_1$, $P_2$) implies the verification of a pair with endpoints ($Q_1$, $Q_2$), then it must be that $Q_1$ is a successor of $P_1$ and $P_2$ is a successor of $Q_2$. %TODO proof required?

Using this information it is possible to choose a minimal subset of path pairs to verify.

We construct a graph with a node for each possible (source, sink) pair in the graph- each node then represents a possible choice for parallel pair endpoint pairs, and greedily search for the smallest set of nodes from which the entire graph would be reachable. The idea is to look for "roots" in the graph that have to be included in the ultimate verification set because they have no predecessor an cannot be verified "through" the verification of some other pair. Then all the successors whose verification is implied by the roots are eliminated.

\begin{verbatim}
# Existng graph
Graph existingGraph;
# New edge
Edge (S, T);

predecessors = S.predecessors(existingGraph)
successors = T.successors(existingGraph)

# Construct the graph
Graph terminalPairGraph;
for q in successors:
    for p in predecessors:
        terminalPairGraph.addNode(q, p)
        # addEdge function should handle addition of edges between nodes that 
        # haven't been defined yet
        for q_pred in q.predecessors(existingGraph):
            for p_succ in p.successors(existingGraph):
                terminalPairGraph.addEdge((q_pred, p_succ))

# Now we reduce the graph.
verificationSet = {}
while (len(terminalPairGraph.nodes) > 0):
    # Choose a random node to start reduction from.
    current_node = terminalPairGraph.nodes[0]
    visited_nodes = set()
    # Climb up the graph till a "root" node is found
    while (len(currentNode.parents()) > 0):
        visited_nodes.add(currentNode)
        # Choose a parent at random.
        currentNode = currentNode.parents()[0]
        # Deal with cycles by reducing them
        if currentNode in visited_nodes:
            edges = getAllEdges(visited_nodes, terminalPairGraph)
            terminalPairGraph.removeNodes(visitedNodes)
            # All the incoming and outgoing edges from the entire visited set are now 
            # transferred to a single node that represents the class
            terminalPairGraph.addNode(currentNode, edges)
    # No predecessor! We should have reached a root!
    verificationSet.add(currentNode)
    terminalPairGraph.removeNodes(currentNode.successors(terminalPairGraph)

# This contains the minimal set of (source, sink) pairs whose verification 
# implies verification of the entire graph. 
return verificationSet

\end{verbatim}
\begin{algorithm}
    \caption{Optimal subset algorithm}
    \label{optimal}
    \KwData{placeholder}
    \KwResult{placeholder}
\end{algorithm}


The result of the algorithm cannot be further reduced. Also, the verification of the parallel pairs returned in the algorithm implies that the output of the previous algorithm must commute, and transitively with Theorem \ref{verifyingSet} that the entire diagram must commute.

The run time of the first step is $O(|V|^4)$, and that of the second step is $O(|V|)$, so that the overall bound is $O(|V|^4)$.

\subsection{Verification}
Ultimately verification is performed by calling an equality oracle (that is beyond the scope of this work) for every comparable pair returned by the previous stage.

\begin{verbatim}
for (path_1, path_2) in parallelPairs:
    if (path_1 != path_2):
        print($"Edge addition does not maintain path independence. 
        Counterexample: {path_1}, {path_2}")
        return False
print("Edge addition maintains path independence.")
return True
\end{verbatim}

\section{Applications}
TODO.
\subsection{Gator}
\subsection{Currency Graph}

To demonstrate our algorithms applied to a real world situation, 
we found inconsistencies in a diagram of the exchange rate between currencies.
Consider a diagram with nodes as currencies and a directed edge being the
conversion rate from its source node's currency to its sink node's currency.
Since the transformation exchange rate of money from any given base currency to a target currency
can be expected to be the same regardless of which intermediate currency transformations
are used, this diagram should commute.

Using an API ~\footnote{https://exchangeratesapi.io}, we built the fully connected diagram of exchange rates between 32 currencies on a given day.
To ensure that it indeed commuted, we started with an empty diagram, and added in edges one by one.
Before the addition of each edge, we used the algorithms (~\ref{polynomial} and ~\ref{optimal}) to ensure the addition of a new edge did not introduce inconsistencies in the existing diagram.
If a new edge was, in fact, problematic, the algorithms returned an example inconsistent
pair that would arise from the addition of the edge.
The pair would consist of two currency transformation sequences with the 
same source currency and ultimate destination currency, but with different effective 
exchange rates values, as computed by taking the product of all the exchange rates
encountered through the chain.

We allowed an "error tolerance" for floating point error.
Graphs were generally found to be consistent up to an error tolerance to the order of magnitude $10^{-6}$.
However, below this threshold we found inconsistencies.
Averaging over evaluation for the first 9 days of 2020, verifying and building a graph to completion (inclusive of the time required by network calls) took 223$\pm$34 seconds using algorithm ~\ref{optimal}, and in 128$\pm$7 seconds with algorithm ~\ref{polynomial}. 
Computation was performed on a Macbook Pro 2015, 2.9 GHz Dual-Core Intel Core i5.
More information about implementation can be found in the evaluation section.

\section{Evaluation}

\section{Appendix: Theorems and additional cases}

Consider all the cycles in the graph that pass through the new edge, unique to their terminal point.
Define a cycle-pair corresponding to cycle c to mean the pair of paths (c, identity on terminal node of c).
Let $C$ be the set of all cycle-pairs (unique to their terminal point) corresponding to cycles that pass through the new edge.

Let V, the set of pairs to verify, be $C \cup R_0$.

\begin{lemma}
\label{one_occurence_lemma}
A path that involves the new edge more than once is equal to a path without multiple occurrences of the edge, under the assumption that the pairs in C are verified.
\end{lemma}
\begin{proof}
Consider the part of the path between the first occurrence of the new edge and the second. This forms a cycle, which has a corresponding check in $C$ and must be verified to be equal to the identity.
This means that the loops containing the multiple occurrences of the new edge may be ignored and the path with a single occurrence of the new edge with the cycles removed is equal to this path.
\end{proof}

\begin{theorem}
\label{verifyingSet}
Verifying that all pairs in V commute implies that all pairs in $R_{all}$ commute.
\end{theorem}
\begin{proof}

Each pair in $R_{all}$ that is not in V would fall into one of these categories:

Case A: pairs that do not involve the new edge: By assumption that the original diagram commutes, these pairs must commute.

Case B: Cases where there is a different pair corresponding to this choice of (source, sink) in V already.
There are four paths to be considered, two from each pair. Those paths which do not include the new edge must already be equal by the assumption that the original diagram commutes.
Those paths which do include the new edge (S,T) can be reduced, by the case for multiple occurrences of a new edge, to a path with a single occurrence of the new edge as described in the preceding lemma \ref{one_occurence_lemma}. These new paths can be divided into three segments: the segment from the source to S, the new edge, and the segment from the last occurrence of T to the sink. The first segments of all paths are equal by assumption that the original diagram commutes. This is also true of the third segment of all the paths. The second segment consists of only the new edge and is the same for all paths. The composition of equal functions is equal, so all the paths passing through the new edge must be equal.
Therefore it is sufficient to check one path that passes through the new edge and one that doesn't. Such a pair belongs to V by construction.
\end{proof}

Modified algorithm for diagrams without the identity:

\begin{verbatim}
Graph existingGraph;
Edge newEdge;

Set parallelPairs = new Set();
for S in existingGraph.Nodes():
    for T in existingGraph.Nodes():
        try:
            Path pathWithNewEdge = 
                FindPath(
                    graph: existingGraph, 
                    sourceNode: S,
                    sinkNode: newEdge.Source()) +
                newEdge as Path +
                FindPath(
                    graph: existingGraph, 
                    sourceNode: newEdge.Sink(), 
                    sinkNode: T)
            if (S==T):
                // Cycles are a special case.
                parallelPairs.add(pathWithNewEdge, pathWithNewEdge+pathWithNewEdge)
                for edge in (S.forwardEdges - pathWithNewEdge[0]):
                    parallelPairs.add(edge, edge+pathWithNewEdge)
                for edge in (S.backwardEdges - pathWithNewEdge[-1]):
                    parallelPairs.add(edge, pathWithNewEdge+edge)
            else:
                Path pathInOldGraph = FindPath(
                    graph: existingGraph, 
                    sourceNode: S, 
                    sinkNode: T);
                parallelPairs.add((pathInOldGraph, pathWithNewEdge));
        catch (pathFindingFailed Exception):
            // No comparable pairs from node {S} to node {T} 
            // that need to be checked
            continue
    cycle = findCycleWithTerminalNode(newEdge.Source()):
    if cycle:    
        parallelPairs.add(newEdge, cycle+newEdge)
    cycle = findCycleWithTerminalNode(newEdge.Source()):
    if cycle:    
        parallelPairs.add(newEdge, newEdge+cycle)
output parallelPairs;
\end{verbatim}

The optional minimization algorithm is modified to exclude nodes representing cycles where the source and sink are equal, since the reduction case no longer will apply in the absence of an implicit identity on the terminal node of the cycle.

\section{Case Studies}

%
% ---- Bibliography ----
%
% BibTeX users should specify bibliography style 'splncs04'.
% References will then be sorted and formatted in the correct style.
%
% \bibliographystyle{splncs04}
% \bibliography{mybibliography}
%
\begin{thebibliography}{8}

\bibitem{commutative} 
Kazuo Murota: {HOMOTOPY BASE OF ACYCLIC GRAPHS - A COMBINATORIAL ANALYSIS OF COMMUTATIVE BY MEANS OF PREORDERED MATROID}. Discrete Applied Mathematics, vol. 17, Issues 1--2, 135--155 (1987).

\end{thebibliography}
\end{document}
