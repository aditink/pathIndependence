\documentclass{article}
\usepackage[utf8]{inputenc}
\usepackage{amsthm}
\usepackage{soul}
\usepackage{tikz}
\usepackage{amssymb}
\usepackage[options ]{algorithm2e}
\usepackage{listings}
\usepackage{xparse}
\usepackage{xcolor}

\newtheorem{lemma}{Lemma}
\newtheorem{theorem}{Theorem}
\newtheorem{definition}{Definition}

\title{Online verification of Commutativity}

\author{ }
\date{April 2020}

\begin{document}

\maketitle

\section{Introduction}

Many systems can be understood as a category, with a collection of sets, and some transformation functions between the sets- for example, in programming languages, type systems with coercion. If a graph is constructed with the sets as nodes and functions as edges between the nodes, then by construction the new graph is commutative. The idea is that each object has a unique representation in a given set in the system, so the transformation mapping between any two nodes should not depend on choice of path used to obtain the transformation.

Since commutative graphs arise in many systems, it is often useful to be able to perform verification and ensure the graphs in question are indeed path independent. 
% As a case study, gator, a domain specific graphics language that allows automatic conversion between geometric spaces internally creates a graph with he transformations between the spaces. 
Naively checking if all paths in a given graph are path independent could require a number of function equality checks as bad as factorial in the number of nodes, since a path consists of an ordering of nodes. 
We present a polynomial time algorithm to verify that commutativity of a graph is maintained over the course of online addition. It uses the minimal possible number of equality checks, since these checks can be expensive. 
To evaluate our results we use two case studies.
First, in the domain specific geometry type language \textit{Gator}, our algorithm is used to ensure that user defined transformations between spaces stay consistent, and the compiler, deterministic as it makes automatic conversion choices. As a second case study, the the algorithm is used to identify inefficiencies in a currency conversion graph.

\section{Problem setup and Terminology}
The closely related problem of identifying the minimal subset of paths to check in order to verify whether a given acyclic graph is commutative, is solved in \cite{commutative}. 
I reproduce relevant terminology and propositions from the paper here for convenience.

First, we formalize the notion of a graph that represents transformations.

We start with a directed graph G=(V,E) with spaces attached to vertices and maps to arcs. Each edge (u, v) in E corresponds to a function that maps elements of u to elements in v. 
This correspondence is stored as a mapping f:$E\rightarrow F$, where F is the semi-group that contains all the functions in the graph, with associative multiplication denoted by $\circ$. f maps each edge to the function it represents.

$P_{all}$ is the set of all paths in G. Since there may be cycles, it may be infinite.

A path is a sequence of edges. The edge-to-function mapping f can be naturally extended to paths. If path p=$e_1\circ e_2 ... e_n$ then $f(p)=f(e_1) \circ f(e_2) ... f(e_n)$.

The notation $\partial^{+}(p)$ is used to indicate that start node pf path p, and similarly $\partial^-(p)$ is the end node of path p. $\partial(p)$ maps to the tuple (start node of p, end node of p).

A pair of paths p and q is called \textit{parallel} iff their terminal nodes are the same, i.e. $\partial(p)=\partial(q)$.

$R_{all}$ is the set of all parallel paths in the graph.

The graph commutes iff f(p)=f(q) $\forall (p,q)\in R_{all}$, which is to say, the compositions of the maps along any path connecting u to v is independent of choice of path.

A bilinking is a parallel pair that is disjoint but for their terminal nodes. The set of all bilinkings is $R_0$.


The main result of \cite{commutative} can be used to verify a graph commutes in the acyclic case. The paper describes the following algorithm to find the ($V^2E$ bounded) minimal set of pairs that needs to be checked.

% Define $r_1>r_2$ for bilinkings $r_1$ = $\{p_l,q_l\}$, $r_2$= $\{p_2, q_2)$ $\in$ $R_0$, if there exists a path p such that $\partial p=\partial r_1$ and p contains $p_2$.
% The $\langle\rangle$ function is defined as:
% $\langle r \rangle = \{ s\in R_0| r>s\}$.

% For bilinking s, let F(s) be the vector in $\mathbb{F}2^{|E|}$ representing the edges present in s (there is a dimension corresponding to each edge, the $n^{th}$ dimension of F(s) is 1 if the corresponding edge is in s, and 0 otherwise). Let this function be extended to sets, so that for some set of bilinkings S, F(S) = $\{ F(s) | s\in S \}$.

% for a set of bilinkings r, the cl function is defined as:
% cl(r) = $\{ s\in R_0|$ s is linearly dependent on $F(r) \}$.

% Using these two function we finally define the function $\sigma$ on a set of bilinkings r as
% $\sigma(r) = \{s \in R_0 | s\in cl(R\cap \langle s \rangle) \}$

An implementation of the described algorithm is

\begin{verbatim}
Graph existingGraph;
R_s = {}
for each node v in V:
    # Let the subsection of the graph reachable from v be S.
    S = existingGraph.extractReachableSection(v)
    # Create a minimum spanning tree of S
    T = createMinimumSpanningTree(S)
    # Find the excluded edges
    excludedEdges = S.edges - T.edges
    # Use these edges to create bilinkings
    for each edge e in excludedEdges:
        firstPath = T.findPath(source: e.source, sink: e.sink)
        R_s.addElement(new Bilinking(firstPath, e)
return R_s
\end{verbatim}

\begin{verbatim}
Find a spanning set Rs, = [r_1, ... ,r_k].

R := Rs,
    for i:=l to K do
        if r_i in sigma(R - r_i) then R:=R - r_i;
return R.

sigma(inputSet S, codomain = Rs):
    output = {}
    for bilinking in Rs:
        smallerPairs = {}
        for candidateBilinking in Rs:
            if exists path(source: bilinking.source, sink:candidateBilinking.source) and
            exists path(source: candidateBilinking.sink, sink: bilinking.sink):
                smallerPairs.add(candidateBilinking)
        # Check if there's a linear dependence between the bilinking and smallerPairs
        # vectorize(pair) returns a vector with a dimension for each edge in the graph,
        # with a 1 if pair includes that edge and 0 if it doesn't.
        matrix = [vectorize(pair) for pair in smallerPair]
        matrix = matrix+[vectorize(bilinking)]
        # if there is a linear dependence
        if determinant(matrix)%2==0:
            output.add(bilinking)
    return output
                
\end{verbatim}

The proof of why this works is complex and the subject of \cite{commutative}, 
% Actually I may be wrong about this
% but at a very high level the cl function captures the redundancy one gets from how "gluing together" bilinkings. For example in the graph below verifying any two of the bilinkings commute implies that the third must too. So cl(any two)=all three bilinkings.

% \begin{center}
% \begin{tikzpicture}[scale=0.2]
% \tikzstyle{every node}+=[inner sep=0pt]
% \draw [black] (15.1,-25.8) circle (3);
% \draw (15.1,-25.8) node {$A$};
% \draw [black] (31.6,-34.7) circle (3);
% \draw (31.6,-34.7) node {$C$};
% \draw [black] (31.6,-18.9) circle (3);
% \draw (31.6,-18.9) node {$B$};
% \draw [black] (47.7,-26.5) circle (3);
% \draw (47.7,-26.5) node {$D$};
% \draw [black] (17.87,-24.64) -- (28.83,-20.06);
% \fill [black] (28.83,-20.06) -- (27.9,-19.9) -- (28.29,-20.83);
% \draw [black] (31.6,-21.9) -- (31.6,-31.7);
% \fill [black] (31.6,-31.7) -- (32.1,-30.9) -- (31.1,-30.9);
% \draw [black] (17.74,-27.22) -- (28.96,-33.28);
% \fill [black] (28.96,-33.28) -- (28.49,-32.46) -- (28.02,-33.34);
% \draw [black] (34.27,-33.34) -- (45.03,-27.86);
% \fill [black] (45.03,-27.86) -- (44.09,-27.78) -- (44.54,-28.67);
% \draw [black] (34.31,-20.18) -- (44.99,-25.22);
% \fill [black] (44.99,-25.22) -- (44.48,-24.43) -- (44.05,-25.33);
% \end{tikzpicture}
% \end{center}

% The $\langle \rangle$ function captures structures like this one, which display the redundancy above. Here, for p=($p_1, p_2), q=(q_1, q_2)$, we have that $p>q$.

% \begin{center}
% \begin{tikzpicture}[scale=0.2]
% \tikzstyle{every node}+=[inner sep=0pt]
% \draw [black] (20.7,-15.7) circle (3);
% \draw (20.7,-15.7) node {$A$};
% \draw [black] (44,-15.7) circle (3);
% \draw (44,-15.7) node {$B$};
% \draw [black] (20.7,-33) circle (3);
% \draw (20.7,-33) node {$A'$};
% \draw [black] (44,-33) circle (3);
% \draw (44,-33) node {$B'$};
% \draw [black] (22.498,-13.306) arc (136.73613:43.26387:13.529);
% \fill [black] (42.2,-13.31) -- (42.02,-12.38) -- (41.29,-13.07);
% \draw (32.35,-8.55) node [above] {$p_1$};
% \draw [black] (41.785,-17.717) arc (-53.13545:-126.86455:15.728);
% \fill [black] (41.79,-17.72) -- (40.85,-17.8) -- (41.45,-18.6);
% \draw (32.35,-21.36) node [below] {$p_2$};
% \draw [black] (23.143,-31.264) arc (120.64183:59.35817:18.065);
% \fill [black] (41.56,-31.26) -- (41.12,-30.43) -- (40.61,-31.29);
% \draw (32.35,-28.24) node [above] {$q_1$};
% \draw [black] (41.52,-34.683) arc (-60.46289:-119.53711:18.601);
% \fill [black] (41.52,-34.68) -- (40.58,-34.64) -- (41.07,-35.51);
% \draw (32.35,-37.6) node [below] {$q_2$};
% \draw [black] (20.7,-18.7) -- (20.7,-30);
% \fill [black] (20.7,-30) -- (21.2,-29.2) -- (20.2,-29.2);
% \draw (20.2,-24.35) node [left] {$\alpha$};
% \draw [black] (44,-30) -- (44,-18.7);
% \fill [black] (44,-18.7) -- (43.5,-19.5) -- (44.5,-19.5);
% \draw (44.5,-24.35) node [right] {$\beta$};
% \end{tikzpicture}
% \end{center}
% $\sigma$ weaves these together to identify, given an input subset, the set of bilinkings that would be commutative should the input subset be verified. 
But at a high level, $\sigma$ identifies reduncancies arising from the structure of the graph and then the approach is to greedily eliminate redundancies.
\cite{commutative} solves the problem of verifying whether a given acyclic diagram is commutative.

In the next section, the distinct problem of verification that a diagram (that may contain cycles) commutes on addition of a new edge is solved. The polynomial time algorithm is found to entail at worst O($|V|^2$) checks and also can be extended to identify the minimum possible set of checks. If the algorithm is run over the course of construction of the graph, it has the same worst case asymptotic runtime O($|V|^2|E|)$ as the one already described.

\section{Solving the online addition problem}

The solution presented here assumes that there is an implicit identity function on every node, as would be the case in most applications. However, a modified version of the algorithm for graphs without this assumption is presented in the appendix.

The first step is to identify all pairs that involve the new edge.
Pairs that do not involve the new edge must already commute by assumption.
Further, for a given (source, sink) pair only a single parallel pair needs to be verified. This is an implication of Theorem \ref{reductionRule}.
Theorem \ref{verifyingSet} shows that should our selected set of pairs along with cycles passing through the new edge be verified commutative, the entire graph must commute. %TODO prove this equivalent?
% \begin{theorem}
% Verifying that all parallel pairs in the graph commute implies that the entire graph must commute.
% \end{theorem}


\begin{verbatim}
Graph existingGraph;
Edge newEdge;

Set parallelPairs = new Set();
for S in existingGraph.Nodes():
    for T in existingGraph.Nodes():
        try:
            Path pathWithNewEdge = 
                FindPath(
                    graph: existingGraph, 
                    sourceNode: S,
                    sinkNode: newEdge.Source()) +
                newEdge as Path +
                FindPath(
                    graph: existingGraph, 
                    sourceNode: newEdge.Sink(), 
                    sinkNode: T)
            if (S==T):
                // Cycles are a speacial case.
                pathInOldGraph = identity(S)
            else:
                Path pathInOldGraph = FindPath(
                    graph: existingGraph, 
                    sourceNode: S, 
                    sinkNode: T);
            parallelPairs.add((pathInOldGraph, pathWithNewEdge));
        catch (pathFindingFailed Exception):
            // No comparable pairs from node {S} to node {T} 
            // that need to be checked
            continue

output parallelPairs;
\end{verbatim}

The try block is executed at most n(n-1)/2 = $O(n^2)$ times where n is the number of nodes.

The bound is asymptotically tight.
This can be seen in the case where the graph contains 2N nodes besides S and T. We consider N of the nodes to be in group 1, and the other N to be in group 2. Every node in group 1 has a forward node to every node in group 2, as well as to S. T has a forward edge to every node in group 2. In this graph, on the addition of edge (S, T), $N^2$ paths need to be verified which is polynomial in 2N+2.

Also notice that if trying to optimize for path length (say, if composing functions is expensive) then "find any path" can be replaced with "find shortest path".

\subsection{(Optional) Optimization Step}

In the case where equality checks are very expensive, we begin by finding the minimal set of (source, sink) pairs such that checking for these pairs logically implies having checked the full graph.

We observe that there are some redundancies in the graph.

Consider the following situation:

\begin{center}
\begin{tikzpicture}[scale=0.2]
\tikzstyle{every node}+=[inner sep=0pt]
\draw [black] (16.6,-34.1) circle (3);
\draw (16.6,-34.1) node {$S$};
\draw [black] (56.8,-33.6) circle (3);
\draw (56.8,-33.6) node {$T$};
\draw [black] (28,-20.1) circle (3);
\draw (28,-20.1) node {$P_1$};
\draw [black] (44.5,-20.1) circle (3);
\draw (44.5,-20.1) node {$P_2$};
\draw [black] (28,-44.9) circle (3);
\draw (28,-44.9) node {$Q_1$};
\draw [black] (44.5,-44.9) circle (3);
\draw (44.5,-44.9) node {$Q_2$};
\draw [black] (31,-20.1) -- (41.5,-20.1);
\fill [black] (41.5,-20.1) -- (40.7,-19.6) -- (40.7,-20.6);
\draw (36.25,-20.6) node [below] {$g_1$};
\draw [black] (26.11,-22.43) -- (18.49,-31.77);
\fill [black] (18.49,-31.77) -- (19.39,-31.47) -- (18.61,-30.84);
\draw (21.74,-25.67) node [left] {$f_1$};
\draw [black] (25.82,-42.84) -- (18.78,-36.16);
\fill [black] (18.78,-36.16) -- (19.01,-37.08) -- (19.7,-36.35);
\draw (23.57,-39.02) node [above] {$f_2$};
\draw [black] (31,-44.9) -- (41.5,-44.9);
\fill [black] (41.5,-44.9) -- (40.7,-44.4) -- (40.7,-45.4);
\draw (36.25,-45.4) node [below] {$g_2$};
\draw [black] (54.59,-35.63) -- (46.71,-42.87);
\fill [black] (46.71,-42.87) -- (47.64,-42.7) -- (46.96,-41.96);
\draw (49.21,-38.76) node [above] {$h_2$};
\draw [black] (54.78,-31.38) -- (46.52,-22.32);
\fill [black] (46.52,-22.32) -- (46.69,-23.25) -- (47.43,-22.57);
\draw (51.19,-25.39) node [right] {$h_1$};
\draw [black] (28,-23.1) -- (28,-41.9);
\fill [black] (28,-41.9) -- (28.5,-41.1) -- (27.5,-41.1);
\draw (27.5,-32.5) node [left] {$l$};
\draw [black] (19.6,-34.06) -- (53.8,-33.64);
\fill [black] (53.8,-33.64) -- (52.99,-33.15) -- (53.01,-34.15);
\draw [black] (44.5,-41.9) -- (44.5,-23.1);
\fill [black] (44.5,-23.1) -- (44,-23.9) -- (45,-23.9);
\draw (45,-32.5) node [right] {$m$};
\end{tikzpicture}
Each arrow represents a path, and (S,T) is the new edge being added.
\end{center}
\begin{theorem}
\label{reductionRule}
If conflicting paths $g_2 = f_2; (S,T); h_2$ then it must be that $g_1 = f_1; (S,T); h_1$.
\end{theorem} 

% monomorphisms

\begin{proof}
We use the fact that $f_1$=$l; f_2$ and $h_1$=$h_2; m$.
\[g_2 = f_2; (S,T); h_2 \Rightarrow l; g_2 = l; f_2; (S,T) ; h_2 \]
\[\Rightarrow l ; g_2 ; m = l ; f_2 ; (S,T) ; h_2 ; m \Rightarrow g_1 = f_1 ; (S,T) ; h_1\]
\end{proof}

Note that the proof is not affected if an of these paths is the identity, eg. if $f_1$ is the identity and S and $P_1$ are actually the same node.

We conclude that verifying a comparable pair of paths with end points ($P_1$, $P_2$) implies the verification of all path pairs ($Q_1$, $Q_2$) such that $Q_1$ is a successor of $P_1$ and $P_2$ is a successor of $Q_2$. A successor S to node N is any node such that there exists a path from N to S. Nodes are also their own successors and predecessors.

Under the assumption that the only path operations allowed are composition and replacement of one path by a different, equal path, as would be true when edges are generic functions, and no other information is available, so that F is a semi-group, this "reduction" rule is also the only rule to reduce the set of path pairs to check by finding implications.

That is to say, if verifying a comparable pair of paths with end points ($P_1$, $P_2$) implies the verification of a pair with endpoints ($Q_1$, $Q_2$), then it must be that $Q_1$ is a successor of $P_1$ and $P_2$ is a successor of $Q_2$. %TODO proof required?

Using this information it is possible to choose a minimal subset of path pairs to verify.

We construct a graph with a node for each possible (source, sink) pair in the graph- each node then represents a possible choice for parallel pair endpoint pairs, and greedily search for the smallest set of nodes from which the entire graph would be reachable.

\begin{verbatim}
# Existng graph
Graph existingGraph;
# New edge
Edge (S, T);

predecessors = S.predecessors(existingGraph)
successors = T.successors(existingGraph)

# Construct the graph
Graph terminalPairGraph;
for q in successors:
    for p in predecessors:
        terminalPairGraph.addNode(q, p)
        # addEdge function should handle addition of edges between nodes that 
        # haven't been defined yet
        for q_pred in q.predecessors(existingGraph):
            for p_succ in p.successors(existingGraph):
                terminalPairGraph.addEdge((q_pred, p_succ))

# Now we reduce the graph.
verificationSet = {}
while (len(terminalPairGraph.nodes) > 0):
    # Choose a random node to start reduction from.
    current_node = terminalPairGraph.nodes[0]
    visited_nodes = set()
    # Climb up the graph till a "root" node is found
    while (len(currentNode.parents()) > 0):
        visited_nodes.add(currentNode)
        # Choose a parent at random.
        currentNode = currentNode.parents()[0]
        # Deal with cycles by reducing them
        if currentNode in visited_nodes:
            edges = getAllEdges(visited_nodes, terminalPairGraph)
            terminalPairGraph.removeNodes(visitedNodes)
            # All the incoming and outgoing edges from the entire visited set are now 
            # transferred to a single node that represents the class
            terminalPairGraph.addNode(currentNode, edges)
    # No predecessor! We should have reached a root!
    verificationSet.add(currentNode)
    terminalPairGraph.removeNodes(currentNode.successors(terminalPairGraph)

# This contains the minimal set of (source, sink) pairs whose verification 
# implies verification of the entire graph. 
return verificationSet

\end{verbatim}

The result of the algorithm cannot be further reduced. Also, the verification of the parallel pairs returned in the algorithm implies that the output of the previous algorithm must commute, and transitively, that the entire graph must commute.
% Here is a naive implemenetation. Note that an important optimization in sparse graphs can be made: Source-sink pairs with no viable adjacent pair should never be added to the graph (as verified by the try-catch block from the first algorithm).

\subsection{Verification}
Ultimately verification is performed by calling an equality oracle (that is beyond the scope of this work) for every comparable pair returned by the previous stage.

\begin{verbatim}
for (path_1, path_2) in parallelPairs:
    if (path_1 != path_2):
        print($"Edge addition does not maintain path independence. 
        Counterexample: {path_1}, {path_2}")
        return False
print("Edge addition maintains path independence.")
return True
\end{verbatim}

\section{Applications}
TODO.
\subsection{Gator}
\subsection{Currency Graph}

\section{Appendix: Theorems and additional cases}
This should be read as a continuation of the notation section, the definitions there will be used.

%Define a cycle-pair to mean the pair (c, c;c) where c is a cycle, which is to say the pair whose first element is a cycle and second element is the cycle traversed twice. *tempted to name this bi-cycle*.

%Let i be the identity function. Define a cycle-identity pair for cycle c to be (c, i). *should I call this unicycle?*

Consider all the cycles in the graph that pass through the new edge, unique to their terminal point.
Define a cycle-pair corresponding to cycle c to mean the pair of paths (c, identity on terminal node of c).
Let $C$ be the set of all cycle-pairs (unique to their terminal point) corresponding to cycles that pass through the new edge.

Let V, the set of pairs to verify, be $C \cup R_0$.

\begin{lemma}
A path that involves the new edge more than once is equal to a path without multiple occurrences of the edge, under the assumption that the pairs in C are verified.
\end{lemma}
\begin{proof}
Consider the part of the path between the first occurrence of the new edge and the second. This forms a cycle, which has a corresponding check in $C$ and must be verified to be equal to the identity.
This means that the loops containing the multiple occurrences of the new edge may be ignored and the path with a single occurrence of the new edge with the cycles removed is equal to this path.
\end{proof}

\begin{theorem}
\label{verifyingSet}
Verifying that all pairs in V commute implies that all pairs in $R_{all}$ commute.
\end{theorem}
\begin{proof}
% We do a kind of "structural induction" to show that verification every pair in $R_{all}$ that aren't in V can wither be reduced down to the verification of smaller components or can be shown to be implied by the verification of V.

Each pair in $R_{all}$ that is not in V would fall into one of these categories:

Case A: pairs that do not involve the new edge: By assumption that the original graph commutes, these pairs must commute.

Case B: Cases where there is a different pair corresponding to this choice of (source, sink) in V already.
There are four paths to be considered, two from each pair. Those paths which do not include the new edge must already be equal by the assumption that the original graph commutes.
Those paths which do include the new edge (S,T) can be reduced, by the case for multiple occurrences of a new edge, to a path with a single occurrence of the new edge as described in the preceding lemma. These new paths can be divided into three segments: the segment from the source to S, the new edge, and the segment from the last occurrence of T to the sink. The first segments of all paths are equal by assumption that the original graph commutes. This is also true of the third segment of all the paths. The second segment consists of only the new edge and is the same for all graphs. The composition of equal functions is equal, so all the paths passing through the new edge must be equal.
Therefore it is sufficient to check one path that passes through the new edge and one that doesn't. Such a pair belongs to V by construction.
\end{proof}

Modified algorithm for graphs without the identity:

\begin{verbatim}
Graph existingGraph;
Edge newEdge;

Set parallelPairs = new Set();
for S in existingGraph.Nodes():
    for T in existingGraph.Nodes():
        try:
            Path pathWithNewEdge = 
                FindPath(
                    graph: existingGraph, 
                    sourceNode: S,
                    sinkNode: newEdge.Source()) +
                newEdge as Path +
                FindPath(
                    graph: existingGraph, 
                    sourceNode: newEdge.Sink(), 
                    sinkNode: T)
            if (S==T):
                // Cycles are a special case.
                parallelPairs.add(pathWithNewEdge, pathWithNewEdge+pathWithNewEdge)
                for edge in (S.forwardEdges - pathWithNewEdge[0]):
                    parallelPairs.add(edge, edge+pathWithNewEdge)
                for edge in (S.backwardEdges - pathWithNewEdge[-1]):
                    parallelPairs.add(edge, pathWithNewEdge+edge)
            else:
                Path pathInOldGraph = FindPath(
                    graph: existingGraph, 
                    sourceNode: S, 
                    sinkNode: T);
                parallelPairs.add((pathInOldGraph, pathWithNewEdge));
        catch (pathFindingFailed Exception):
            // No comparable pairs from node {S} to node {T} 
            // that need to be checked
            continue
    for cycle in findCyclesWithTerminatNode(newEdge.Source()):
        parallelPairs.add(newEdge, cycle+newEdge)
    for cycle in findCyclesWithTerminatNode(newEdge.Source()):
        parallelPairs.add(newEdge, newEdge+cycle)
output parallelPairs;
\end{verbatim}

The optional minimization algorithm is modified to exclude nodes representing cycles, since the reduction case no longer will apply in the absence of an implicit identity on the terminal node of the cycle.
*I think the minimality of minimized subset still holds, but not 100\% sure*

\section{Case Studies}

\begin{thebibliography}{9}
\bibitem{commutative} 
Kazuo Murota. 
\textit{HOMOTOPY BASE OF ACYCLIC GRAPHS - A COMBINATORIAL ANALYSIS OF COMMUTATIVE BY MEANS OF PREORDERED MATROID}. 
Discrete Applied Mathematics, 1987.

\end{thebibliography}
\end{document}